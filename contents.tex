\title{%
  Privacy-enhancing technologies and surveillance in the digital
  society. Position paper and discussion prompt.
}
\date{}
\author{Daniel Bosk}
\author{Sonja Buchegger}
\institute{%
  KTH Royal Institute of Technology\\ EECS School of Electrical
  Engineering and Computer Science\\
  \email{\{dbosk,buc\}@kth.se}
}

\mode<article>{\maketitle}
\mode<presentation>{%
  \begin{frame}
    \maketitle
  \end{frame}
}

\mode*

%\begin{abstract}
%  \input{abstract.tex}
%\end{abstract}


%\section{Introduction}

%\paragraph{What is the problem?}

\begin{abstract} 
The field of \acp{PET}, a subfield of Computer Science, aims at
developing tools to counter surveillance and to give people control over
their data, usually by cryptographic means. There are numerous
\acp{PET} available that would make digital surveillance more
difficult, but few of them are widely used. We list some of them along
with some others that have come out of research in the past or are
active research areas, both from the \acp{PET} community at large and
our own research group, to give a flavor of the technical
possibilities. We conclude with a discussion prompt on the mismatch
between available and used \acp{PET} for surveillance resistance. This
lack of technology transfer is especially grave when considering the
increase in surveillance and privacy breaches in recent years.
Insights will help \acp{PET} researchers  to 
adapt their methodology and the tools they produce to get more 
wide-spread adoption.
\end{abstract}

%\paragraph{Why is it a problem?}



%\paragraph{Why is this important?}



\section{Examples of Deployed \acp{PET}}

WhatsApp started to introduce \ac{E2E} encryption for its users in
2014~\cite{WhatsAppIntroducesE2Eencryption} (completed in
2016~\cite{WhatsAppE2Ecomplete}).  It seems that this was more the
company's political stance rather than user demand (TextSecure, now
Signal, has been around since 2010).

This prevents governments (\ie service providers) from reading the
contents of users' messages.  It still, however, allows WhatsApp, and
now Facebook, to see metadata, \ie who communicates with whom and
when. Much information can be inferred from
metadata~\cite{DevilInMetadata}.  Some of the metadata can also be
inferred through traffic analysis by \eg Internet and telephony
service providers.


Tor~\cite{Tor}, on the other hand, also hides metadata.
It is mainly known in the media for the \enquote{Dark Web}%~\cite{?} 
and its main use comes through the Tor Browser.%~\cite{?}.
The Tor Browser additionally blocks trackers, \enquote{Like Buttons} (\ie more 
trackers) and other elements that can be used for tracking (\eg browser 
fingerprinting).
This prevents web sites from tracking users.

However, Tor Browser cannot prevent the provider (and indirectly surveillance), \eg Facebook, 
from collecting data if the user is logged in. Some of the better
known Facebook alternatives that attempt to solve this problem by
some degree of decentralization are  diaspora*%~\cite{?}.
and mastodon. 
%All data is private to the individual, it tries to break the centralization 
%which caused the metadata problems above, but it is not fully decentralized.

Cryptocurrencies such as bitcoin are an attempt at coming closer to
anonymous e-cash. PGP (Pretty Good Privacy) allows users to send and
receive encrypted e-mail.

\section{\acp{PET} Research}

Anonymous credentials are a prime example of available \acp{PET},
even backed by industry (with similar approaches from IBM and
Microsoft), that nevertheless have not been deployed. The idea behind
anonymous credentials is simple: we often share more details and
personally identifying information than needed for a particular
situation or transaction. Anonymous credentials make use of
zero-knowledge proofs to allow people to provide proof of necessary
conditions, \eg that they have an account with a company they want to
stream a move from, that they are old enough for a particular movie
rating, that they have a valid credit card, and that they have paid
for their subscription, all without revealing unnecessary particulars
such as their name, date of birth, credit card number. 

Besides cryptography, another promising approach for privacy
protection and surveillance resistance is decentralization. 
While decentralization removes the best position for surveillance for one entity, \eg 
a social network service provider such as Facebook, it introduces new problems.
In a decentralized system, all transactions are made in public, this means that 
suddenly everyone is in Facebook's position.
This must be considered when designing such systems.
We will now present some examples from our work.

One example of problems is access control.
As an example, Alice and Bob are friends on a decentralized social media 
platform.
Whenever Alice posts a photo or message, it is published in such a way that her 
friends know where to find it and they can decrypt, but no one else can.
Then, whenever Bob is no longer able to decrypt he can infer that Alice has 
removed him.
\Textcite{PPACinPubFS} analysed how access-control policy changes could leak in 
various ways through metadata available in such decentralized systems and 
suggested a way to reduce these leaks, in particular, to make such changes 
without Bob learning about it.
(This also implies that no one but Alice's friends can learn whenever Alice 
posts.)

Another example is sharing calendars and scheduling events.
\Textcite{EventsInvitations} proposed a protocol for inviting to and responding 
to invites for events in a privacy-preserving manner.
This allowed organizers to set varying levels of privacy, \eg that invitees 
cannot see who else is invited until they accept the invitation.

Bosk analysed the privacy implications of calendar sharing for scheduling 
events without active involvement of all people.
This allowed Bosk, Buchegger and Wåreus to propose designs where leaks are 
limited, \eg by rate limiting and differential privacy.

As a final example, Bosk, Bouget, Buchegger and Gambs provided a protocol for 
privacy-preserving, yet verifiable crowd counting and petitions.
To some extent, this protocol combines location-based services and electronic 
voting.
(The location is not relevant in the case of petitions.)
This allows supporters of a cause to have a demonstration and to provide 
verifiable proof of the participation to third parties.
These proofs are unlinkable to a participant's identity, yet the participant 
cannot produce two proofs to be counted twice.
The latter is needed to for the correctness of the count, the former to protect 
against someone using a participant's participation against them.

\section{Discussion Prompt}
Although legions of Computer Scientists have been developing
\acp{PET}, only a small fraction of the results are widely known in
the general population and only few of these actively used by
significant parts of the population. 

Clearly, there is a mismatch between what Computer Scientists think
people should do to protect against surveillance by government or
commercial entities and reality.

Whether it is a case of engineer's disease providing technical
solutions to people problems, lack of usability considerations,
negative impact on the performance of the privacy-enhanced service,
apathy, lack of communication, solving the wrong problems, the network
effect not being reached, perceptions and beliefs or fear of raising
suspicions, or any other reason or combination of factors, is
unclear. This implies that it is also unclear what could/should be
done to change that.

Input and insights from other disciplines concerned with surveillance
could help inform research on \acp{PET} to make it more relevant,
effective, and actually used to counter surveillance. This is
especially important in this area since many privacy properties depend
on a critical mass of users~\cite{AnonymityLovesCompany} and in an era
where examples like Cambridge 
Analytica~\cite{%
  cambridge-analytica-wired,
  cambridge-analytica-guardian,
  cambridge-analytica-nytimes,
  cambridge-analytica-wp%
} illustrate how data collection and analysis can be used for the
purpose of manipulation of millions of users. 



% Why are they not used more and how can we change that?

% For instance: Why do people not encrypt their emails?
% Is it usability, does it introduce problems, or is it simply the network effect 
% (no one else does it)?

% Why do people not browse the web anonymously with Tor?
% Is it usability, \ie too slow, makes web sites unusable (blocking some active 
% contents)?
% Or do people not see a problem?

% Or do these technologies solve the wrong problem?

%%% REFERENCES %%%

\begin{frame}[allowframebreaks]
  \printbibliography
\end{frame}
