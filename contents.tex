\title{%
  Privacy-enhancing technologies and surveillance in the digital society
}
\author{Daniel Bosk}
\author{Sonja Buchegger}
\institute{%
  KTH EECS
}

\mode<article>{\maketitle}
\mode<presentation>{%
  \begin{frame}
    \maketitle
  \end{frame}
}

\mode*

%\begin{abstract}
%  % What's the problem?
% Why is it a problem? Research gap left by other approaches?
% Why is it important? Why care?
% What's the approach? How to solve the problem?
% What's the findings? How was it evaluated, what are the results, limitations, 
% what remains to be done?

% XXX Summary
\emph{Summary:}
\dots

% XXX Motivation and intended learning outcomes
\emph{Intended learning outcomes:}
\dots

% XXX Prerequisites
\emph{Prerequisites:}
\dots

% XXX Reading material
\emph{Reading:}
\dots

%\end{abstract}


\section{Introduction}

\paragraph{What is the problem?}

There are numerous \acp{PET} out there that would make digital surveillance 
more difficult.
Why are they not used more and how can we change that?

For instance: Why do people not encrypt their emails?
Is it usability, does it introduce problems, or is it simply the network effect 
(no one else does it)?

Why do people not browse the web anonymously with Tor?
Is it usability, \ie too slow, makes web sites unusable (blocking some active 
contents)?
Or do people not see a problem?

Or do these technologies solve the wrong problem?

\paragraph{Why is it a problem?}

The answer to these questions will help researchers in the area of \acp{PET} to 
adapt their research and the tools they produce so that they get more 
wide-spread adoption.
This is especially important in this area since many privacy properties depend 
on many users~\cite{AnonymityLovesCompany}.

\paragraph{Why is this important?}

The importance of this is illustrated by examples like Cambridge 
Analytica~\cite{%
  cambridge-analytica-wired,
  cambridge-analytica-guardian,
  cambridge-analytica-nytimes,
  cambridge-analytica-wp%
}, where the data from the surveillance apparatus was used to manipulate entire 
populations.


\section{Deployed \aclp{PET}}

WhatsApp started to introduce \ac{E2E} encryption for its users in 
2014~\cite{WhatsAppIntroducesE2Eencryption} (completed in 
2016~\cite{WhatsAppE2Ecomplete}).
It seems that this was more the company's political stance rather than user 
demand (TextSecure, now Signal, has been around since 2010).

This prevents governments (\ie service providers) from reading the contents of 
users' messages.
But it still allows WhatsApp, and now Facebook, to see metadata, \ie who 
communicates with whom and when.
Some of this can also be inferred through traffic analysis by \eg Internet and 
telephony service providers.
Much information can be inferred from metadata~\cite{?}.

Tor~\cite{Tor}, on the other hand, also hides metadata.
Its mainly known in the media for the \enquote{Dark Web}~\cite{?} and its main 
use comes through the Tor Browser~\cite{?}.
Tor Browser additionally blocks trackers, \enquote{Like Buttons} (\ie more 
trackers) and other elements that can be used for tracking (\eg browser 
fingerprinting).
This prevents web sites from tracking users.

However, Tor Browser cannot prevent the surveillance apparatus, \eg Facebook, 
from collecting data if the user actively uses Facebook.
One of the more known Facebook alternatives that comes the closest to solving 
the problem is diaspora*~\cite{?}.
All data is private to the individual, it tries to break the centralization 
which caused the metadata problems above, but it is not fully decentralized.


\section{Works in progress}

Deniable messaging
\dots

Calendar sharing and scheduling
\dots

Privacy-preserving, yet verifiable crowd counting and petitions
\dots


%%% REFERENCES %%%

\begin{frame}[allowframebreaks]
  \printbibliography
\end{frame}
